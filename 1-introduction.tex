% !TEX root = main.tex
\section{Introduction}
\subsection{Static Analysis}
\subsection{Description of CodeQL}
% Quick history Overview

\subsection{Architecture Outline}
% TODO check again
% overview paragraph is not very good/precise
In this section, we will give a short overview of the CodeQL architecture, as well as where our
project fits in. We describe the extraction process where source code is turned into a relational database,
as well as the QL language and the libraries supporting the queries.

In order to analyze source code, it first needs to be processed into a database representation.
For each language an \emph{extractor} parses the source code and outputs a database file.
The extractor is language specific and usually hooks into a (possibly modified) compiler for that language.
The output of the extractor is a relational database in a language-specific \emph{database schema}.
This database schema evolves along with the language, as well as extractor capabilities.
For example, the Java database schema changed when preliminary support for the Java 13 yield statement was
added to the extractor.

The \emph{database engine} allows users to query the database file produced by the extractor.
Queries are written in the QL language.
Both the database engine and the QL language are agnostic to the language the analyzed source code 
is written in.
This allows developers at GitHub with relative easy to add support for new languages to the CodeQL ecosystem.
In theory, QL queries are also independent of the language the analyzed source code is written.
However, in practice queries are always tied intimately to the language of the analyzed source code,
as they depend on the language-specific database schema.
Furthermore, queries can't gloss over language-specific semantics either {---} one example of many would be
that signed integer overflow needs to be treated differently in Java than in C/C++.

For each language, a set of \emph{libraries} supports the development of queries.
These libraries are implemented in QL.
The libraries implement features such as taint tracking, dataflow analysis or control flow graphs.
For every supported language by the CodeQL analyzers, developers need to re-implement all of these libraries from
scratch, as components can't be shared across languages.
The only exception is the dataflow library, which is separated into a language-agnostic and 
language-specific part.
This language-agnostic part of the dataflow library is shared between the libraries for Java, C/C++ and C\#.

Using those libraries, developers at GitHub write \emph{queries} to detect bugs in the analyzed program.
For example, one query detects when variable dereference in Java might be null.
Another query detects if user input ends up in a database query, without the user input being 
sanitized properly for use in such a query.
Queries often rely on the supporting libraries to do the heavy lifting, e.g.\ the SQL injection detection 
is built on top of the dataflow library.


% TODO
The queries are accessible to researchers in academia and all open-source projects free of charge 
via the \href{https://LGTM.com}{LGTM.com} online platform and the CodeQL CLI utility. % TODO font for LGTM
Bugs can be reported via the issue tracker and the project takes pull requests.


All queries as well as the supporting libraries are open-source and published 
on GitHub\footnote{\url{https://github.com/Semmle/ql}}.
The extractors and the query engine are available only in compiled form under a 
proprietary license\footnote{\url{https://github.com/github/codeql-cli-binaries}}.
While the license is quite restrictive,
usage of the CodeQL CLI utility for academic research is explicitly allowed.

Our project outside the course scope contributes improvements to the dataflow library,
both the shared and the language-specific part. We only implemented the language-specific 
functionality, and an GitHub employee provided stubs for C/C++ and C\#.
The individual features are discussed in the next chapters.

% Every language has its own extractor, which parses source code into the language-specif%ic database scheme.
% TODO? This database scheme evolves together with the language and extractor, 
%The queries and the supporting libraries are all specific to the language being analyzed,
%only the evaluator

\subsection{Philosophy}
Using static analysis for bug finding is lacking in the theoretical foundation {---} 
analyses are usually neither sound nor complete.
This means that a query can report \emph{false positives}, problems flagged by the query that 
are not bugs at all.
Furthermore, queries can also not report instances of a bug which are present in the code,
that instance is then called a \emph{false negative}.
Obviously both false positives and negatives are not desirable, but static analyzers go a long way
to reduce false positives as a means to reduce noise.
If false positives are reduced, almost every problem flagged by an analysis is a real bug.
This makes the static analyzer a very useful tool.
% TODO reword
% TODO general thing

\cite{qlpaper}
\subsection{Evaluation Model}
Every predicate is computed bottoms-up, so when a predicate is encountered, its
tuple set is computed.
Recursive predicates are computed as fixed-point iterations, which means that the database engine 
first adds the tuple sets arising from the base cases, and then from the recursive cases, until
no more tuples can be added. (TODO improve).
Mutual recursions are TODO.

Another useful feature of QL is that it includes syntactic sugar to compute the reflexive-transitive
and the transitive closure of predicates.
For example, a predicate \texttt{getSucc} on control flow nodes gives the set of all reachable
nodes by invoking \texttt{getSucc*}.

\subsection{QL And Relational Database TODO title}
At the heart of QL lives a relational database, and QL is designed to query a relational database.
Every predicate, the basic notion of a BLA, can be regarded as a tuple.
