% !TEX root = main.tex
\section{Formalization of Dataflow with dIMP}

In this section, we are going to formalize what we mean by dataflow,
give a algorithm specification for computing dataflow and prove that any algorithm conforming
to that algorithm specification is a sound dataflow algorithm.

We use the toy language IMP as presented in~\cite{sat}.
The lecture notes themselves heavily borrow from~\cite{fsopl} for the
presentation and semantics of IMP.

\subsection{Syntax of dIMP}
We extend the syntax of IMP a bit
and call the resulting language dIMP (for dataflow IMP):
\begin{align*}
    n \in  \:       & \Z\\
    \flaggedn{} \in \: & \StoreDomain = \{\flagged{} \mid n \in \Z\}\\
    X \in  \:       & \Loc = \{x, y, z, sum, \ldots\}\\
    a \Coloneqq \: &\literalint{} \mid X \mid a_0+a_1 \mid a_0-a_1 \mid a_0 \times a_1 \mid \defaultsource\\
    t \Coloneqq \: & \btrue \mid \bfalse                                                \\
    b \Coloneqq \: & t \mid a_0 = a_1 \mid a_0 \leq a_1 \mid \neg b_0 \mid b_0 \land b_1 \\
    c \Coloneqq \: &\skipcmd \mid \defaultstore \mid \defaultseq \mid \defaultif \mid \defaultwhile \mid \defaultsink\\
    f \Coloneqq \: & \textbf{c} \mid \textbf{t}\\
\end{align*}
Stores are denoted with $\store \in \Sigma = \Loc \to \StoreDomain$.
Stores are function from locations to the set of tagged integer
values.

The intuition here is that for all $n \in \Z$, 
each value is tagged by either \textbf{c} for clean or \textbf{t} for being tracked
as having dataflow.
If the tag is irrelevant in the current context, we write $\flaggedn{}$ for a tagged
value.
A tracked value originates from a source, and is preserved by value-preserving operations.
The command $\defaultsink$ is a special marker for the dataflow sink.
It aborts the program if a tracked value reaches the sink, thus indicating a policy 
violation.
In real programs, the dataflow source would likely be a procedure reading input,
i.e.\ from a terminal, and the dataflow sink would likely be a procedure 
printing the data to a terminal, or otherwise communicating with the outside world.

\iffalse
There is a new judgement of the form 
\fbox{$\bigstep{a}{\sigma}{\flaggedn{}}$}:
\begin{align*}
    &\bsrule{EA-Num}{}{\bigstep{\literalint}{\store}{\clean}} \qquad
    \bsrule{EA-Loc}{}{\bigstep{X}{\store}{\store(X)}}\qquad
    \bsrule{EA-Plus}
    {\bigstep{a_0}{\store}{\flagged[n_0][0]}
    \qquad \bigstep{a_1}{\store}{\flagged[n_1][1]}}
    {\bigstep{a_0+a_1}{\store}{\clean[(n_0+n_1)]}}\\
    &\bsrule{EA-Minus}
    {\bigstep{a_0}{\store}{\flagged[n_0][0]}
    \qquad \bigstep{a_1}{\store}{\flagged[n_1][1]}}
    {\bigstep{a_0-a_1}{\store}{\clean[(n_0-n_1)]}} \qquad
    \bsrule{EA-Times}
    {\bigstep{a_0}{\store}{\flagged[n_0][0]}
    \qquad \bigstep{a_1}{\store}{\flagged[n_1][1]}}
    {\bigstep{a_0 \times a_1}{\store}{\clean[(n_0 \times n_1)]}}\\
    &\bsrule{EA-Source}{\bigstep{a}{\store}{\flagged}}
    {\bigstep{\source{a}}{\store}{\tracked}}
\end{align*}

The Judgement $\langle c, \sigma \rangle \downarrow \sigma'$ is replaced by 
$\bigstep{c}{\sigma}{\excstore'}$
where $\excstore{}: (\Loc \to \StoreDomain) \cup \{\exception\}$.
The state $\exception$ indicates that the program was aborted due to a policy violation.
\\
Judgement \fbox{$\bigstep{c}{\sigma}{\excstore'}$}:
\begin{align*}
    &\bsrule{EC-Skip}{}{\bigstep{\skipcmd}{\store}{\store}}\qquad
    \bsrule{EC-Assign}{\bigstep{a}{\store}{\flaggedn{}}}
    {\bigstep{\defaultstore}{\store}{\store{}[X \mapsto \flaggedn{}]}}\\
    &\bsrule{EC-Seq}{\bigstep{c_0}{\sigma}{\store''} \qquad \bigstep{c_1}{\store''}{\excstore'}}
    {\bigstep{\defaultseq}{\store}{\excstore'}}\qquad
    \bsrule{EC-SeqAbrt}{\bigstep{c_0}{\sigma}{\exception}}
    {\bigstep{\defaultseq}{\store}{\exception}} \\
    &\bsrule{EC-IfT}{\bigstep{b}{\store}{\btrue}\qquad \bigstep{c_0}{\store}{\excstore'}}
    {\bigstep{\defaultif}{\store}{\excstore'}} \qquad
    \bsrule{EC-IfF}{\bigstep{b}{\store}{\bfalse}\qquad \bigstep{c_1}{\store}{\excstore'}}
    {\bigstep{\defaultif}{\store}{\excstore'}}\\
    &\bsrule{EC-WhileF}{\bigstep{b}{\sigma}{\bfalse}}
    {\bigstep{\defaultwhile}{\store}{\store}} \\
    &\bsrule{EC-WhileT}{\bigstep{b}{\sigma}{\btrue} \qquad \bigstep{c_0}{\store}{\store''}\qquad
    \bigstep{\defaultwhile}{\store''}{\excstore'}}
    {\bigstep{\defaultwhile}{\store}{\excstore'}}\\
    &\bsrule{EC-WhileTAbrt}{\bigstep{b}{\sigma}{\btrue} \qquad \bigstep{c_0}{\store}{\exception}}
    {\bigstep{\defaultwhile}{\store}{\exception}}\\
    &\bsrule{EC-Sink}{\bigstep{a}{\store}{\clean{}}}
    {\bigstep{\defaultsink}{\store}{\store}} \qquad
    \bsrule{EC-SinkAbrt}{\bigstep{a}{\store}{\tracked{}}}
    {\bigstep{\defaultsink}{\store}{\exception}}\\
\end{align*}

The judgement operations for Boolean expressions are modified to ignore the tracked-flag and work
just as in IMP. This is analogous to how $\textsc{EA-Plus}$ etc.\ work.
\fi

\subsection{Syntax and Semantics of SSA-dIMP}
To facilitate the dataflow analysis, we describe an SSA form on dIMP with a concrete
syntax and semantics.
We do not describe how to transform a dIMP program to an SSA-dIMP program, but 
refer the reader to the literature, where different SSA construction algorithms are
described. Some even come with computer-verified correctness and minimality guarantees,
we refer the reader to~\cite{verifiedssa}.

The syntax for Boolean and arithmetic expressions in SSA-dIMP is the same as in dIMP.
However, the command syntax changes.
We now denote commands by $s$ (short for statement, or SSA-command) to highlight the differences
from the commands denoted by $c$ in dIMP.
\begin{align*}
    s \Coloneqq &\:\skipcmd \mid \defaultstore \mid \defaultssaseq \mid \defaultssaif\\
    &\mid \defaultssawhile  \mid \defaultsink \\
    \philist \Coloneqq &\: \defaultphilistn\\
\end{align*}
At every point where $\varphi$-nodes can placed in the control flow graph
of a dIMP program, the syntax of SSA-dIMP has an explicit list of $\varphi$-nodes.

\subsubsection*{The SSA Type System}
A program in SSA-dIMP is only valid if it conforms to the SSA properties.
To enforce that, we describe a simple type system.
Every well-typed expression has type $\tint$, thus the set of all types is $\Tau = \{ \tint \}$.
Let $\Gamma, \Delta: \Loc \partialf \Tau$ be partial maps with finite domain.
These are the type contexts.
We define the join operator $\join$ on $\Tau$ by $\tint \join \tint = \tint$.
Furthermore, we will use $\dunion$ to be the union of maps, 
where we assert that the domains of the maps are disjoint.

% TODO weird sentence!
Then we proceed to show that these sets are enough to prove that the program is indeed in SSA form.
The easiest way to get these sets is to run an SSA construction algorithm on a dIMP program.
It is a gap in the provided proofs that we do not show that every dIMP program
can be transformed in a valid SSA-dIMP program (especially, we omit showing that
$\varphi$-nodes will be placed exactly at the places they are allowed in the syntax
of SSA-dIMP).

The type system has the following judgments:
\\
Judgement \fbox{$\typestep{\Gamma}{a}{\tint}$}:
\begin{align*}
    &\trule{SSA-Num}{}{\typestep{\Gamma}{\overline{n}}{\tint}}\qquad
    \trule{SSA-Loc}{}{\typestep{\Gamma}{X}{\Gamma(X)}}\\
    &\trule{SSA-Plus}{\typestep{\Gamma}{a_0}{\tint}\qquad\typestep{\Gamma}{a_1}{\tint}}
    {\typestep{\Gamma}{a_0+a_1}{\tint}}\qquad
    \trule{SSA-Minus}{\typestep{\Gamma}{a_0}{\tint}\qquad\typestep{\Gamma}{a_1}{\tint}}
    {\typestep{\Gamma}{a_0-a_1}{\tint}}\\
    &\trule{SSA-Times}{\typestep{\Gamma}{a_0}{\tint}\qquad\typestep{\Gamma}{a_1}{\tint}}
    {\typestep{\Gamma}{a_0 \times a_1}{\tint}}\qquad
    \trule{SSA-Source}{\typestep{\Gamma}{a}{\tint}}{\typestep{\Gamma}{\source{a}}{\tint}}\\
\end{align*}
Judgement \fbox{$\typestmt{\Gamma}{b}$}:
\begin{align*}
    &\trule{SSA-Bool}{}{\typestmt{\Gamma}{t}}\qquad
    \trule{SSA-Eq}{\typestep{\Gamma}{a_0}{\tint}\qquad \typestep{\Gamma}{a_1}{\tint}}
    {\typestmt{\Gamma}{a_0 = a_1}}\\
    &\trule{SSA-Leq}{\typestep{\Gamma}{a_0}{\tint}\qquad \typestep{\Gamma}{a_1}{\tint}}
    {\typestmt{\Gamma}{a_0 \leq a_1}}\qquad
    \trule{SSA-Neg}{\typestmt{\Gamma}{b}}
    {\typestmt{\Gamma}{\neg b}}\qquad
    \trule{SSA-And}{\typestmt{\Gamma}{b_0}\qquad\typestmt{\Gamma}{b_1}}
    {\typestmt{\Gamma}{b_0 \land b_1}}\\
\end{align*}
Judgement \fbox{$\typestep{\Gamma;\Delta_0;\Delta_1}{\philist}{\Delta}$}:
\begin{align*}
    &\trule{SSA-$\varphi$}{}{\typestep{\Gamma;\Delta_0;\Delta_1}{\defaultphilistn}{[X_1 \mapsto \type_1, \ldots, X_n \mapsto \type_n]}}\\
    &\text{(if for $i \in \{1, \ldots, n\}: X_i \notin \dom\Gamma \land \type_i = (\Gamma \dunion \Delta_0)(Y_i) \join (\Gamma \dunion \Delta_1)(Z_i)
    \land \forall j \neq i: X_j \neq X_i$)}
\end{align*}
Note that this implies that this rule is only applicable if $Y_i$, $Z_i$ exist in the respective domains
of the partial functions.
\\
Judgement \fbox{$\typestep{\Gamma}{s}{\Delta}$}:
\begin{align*}
    &\trule{SSA-Skip}{}{\typestep{\Gamma}{\skipcmd}{\emptyset}}\qquad
    \trule{SSA-Assign}{\typestep{\Gamma}{a}{\tint}}{\typestep{\Gamma}{\defaultstore}{[X \mapsto \tint]}} \text{($X \notin \dom\Gamma$)}\\
    &\trule{SSA-Seq}{\typestep{\Gamma}{s_0}{\Delta_0}\qquad
    \typestep{\Gamma \dunion \Delta_0}{s_1}{\Delta_1}}{\typestep{\Gamma}{\defaultssaseq}{\Delta_0 \dunion \Delta_1}}\\
    &\trule{SSA-If}{\typestmt{\Gamma}{b}\qquad\typestep{\Gamma}{s_0}{\Delta_0}\qquad \typestep{\Gamma}{s_1}{\Delta_1}\qquad
    \typestep{\Gamma;\Delta_0;\Delta_1}{\philist}{\Delta}}
    {\typestep{\Gamma}{\defaultssaif}{\Delta}}\\
    &\trule{SSA-While}{\typestep{\Gamma;\emptyset;\Delta_1}{\philist}{\Delta}\qquad
    \typestmt{\Gamma \dunion \Delta}{b}\qquad\typestep{\Gamma \dunion \Delta}{s_0}{\Delta_1}}
    {\typestep{\Gamma}{\defaultssawhile}{\Delta}}\\
    &\trule{SSA-Sink}{\typestep{\Gamma}{a}{\tint}}{\typestep{\Gamma}{\sink{a}}{\emptyset}}\\
\end{align*}

Note that by this definition, the first argument of $\varphi$-nodes for \textbf{if} commands
corresponds to the variable after executing the \btrue{} branch, and the second argument to that
of the variable after executing the \bfalse{} branch.
For \textbf{while} commands, the first argument to the $\varphi$-node corresponds to the value
before executing $s_0$ the first time, and the second argument corresponds to the value 
after executing the loop body.
The sets $\Gamma$ and $\Delta$ are natural outputs of an algorithm that converts a 
dIMP program into SSA-dIMP form.
They prove that a (not necessarily minimal) SSA form was indeed achieved.
Inputs to the program are provided in the initially provided store, and $\Gamma$
has to contain all variables accessed by the program that don't have a definition in the program.

\subsubsection*{The Semantics of SSA-dIMP}
Stores are now partial functions $\store : \Sigma \partialf \Loc$ to indicate that
variables have to be defined before they can be read.
When updating a store, we now use the syntax 
$\store \dunion [X \mapsto \flagged{}]$ to indicate that the store $\store$ does not
contain the variable $X$.
\autoref{thm:ssa-gamma-delta-disjoint} in combination with ~\autoref{thm:ssa-progress}
justify this notation.

The judgement for Booleans and arithmetic expressions does not change.
As before, we use $\excstore{}$ to denote stores that represent an \exception{}
state or a partial function, i.e.\ $\excstore{} \in (\Loc \partialf \StoreDomain) \union \{\exception\}$
For $\varphi$-nodes, we introduce two judgements --- one that evaluates all the variable definitions
to the first argument, and one that evaluates them all to the second argument.
Both judgements are provided with a filter $\filter \subseteq \dom\sigma$ that specifies which variables 
from the outer store should be kept.
The symbol $\store\restrict{\filter}$ denotes the restriction of the map $\store$
to the domain $\filter$.
In order to correctly model the execution of $\varphi$-nodes for \textbf{while} commands,
a second judgement is set up.
The regular command judgement only populates the variables defined by the $\varphi$-nodes
and then defers to a special while-command judgement that evaluates all loop executions,
evaluating the $\varphi$-nodes with the store after the loop body execution.
\\
Judgement \fbox{$\bigsteppl{\philist}{\filter}{\store}{\store'}$}:
\begin{align*}
    &\bsrule{E$\Phi_1$-Assign}{}
    {\bigsteppl{\defaultphilistn}{\filter}{\store}{\storeassign{\store\restrict{\filter}}{X_1 \mapsto \store(Y_1),\ldots, X_n \mapsto \store(Y_n)}}}\\
\end{align*}
Judgement \fbox{$\bigsteppr{\philist}{\filter}{\store}{\store'}$}:
\begin{align*}
    &\bsrule{E$\Phi_2$-Assign}{}
    {\bigsteppr{\defaultphilistn}{\filter}{\store}{\storeassign{\store\restrict{\filter}}{X_1 \mapsto \store(Z_1),\ldots, X_n \mapsto \store(Z_n)}}}\\
\end{align*}
\\
Judgement \fbox{$\bigstepw{s}{\filter}{\store}{\excstore'{}}$}:
\begin{align*}
    &\bsrule{EW-WhileF}{\bigstep{b}{\store}{\bfalse}}
    {\bigstepw{\defaultssawhile}{\filter}{\store}{\store}}\qquad\\
    &\bsrule{EW-WhileTAbrt}{\bigstep{b}{\store}{\btrue} \qquad 
    \bigstep{s_0}{\store}{\exception}}
    {\bigstepw{\defaultssawhile}{\filter}{\store}{\exception}}\\
    &\bsrule{EW-WhileT}{\bigstep{b}{\store}{\btrue} \enskip
    \bigstep{s_0}{\store}{\store''} \enskip
    \bigsteppr{\philist}{\filter}{\store''}{\store'''} \enskip
    \bigstepw{\defaultssawhile}{\filter}{\store'''}{\excstore'}}
    {\bigstepw{\defaultssawhile}{\filter}{\store}{\excstore'}}
\end{align*}
\\
Judgement \fbox{$\bigstep{s}{\store}{\excstore{}}$}:
\begin{align*}
    &\bsrule{EC-Skip}{}{\bigstep{\skipcmd}{\store}{\store}}\qquad
    \bsrule{EC-Assign}{\bigstep{a}{\store}{\flaggedn{}}}
    {\bigstep{\defaultstore}{\store}{\storeassign{\store}{X \mapsto \flaggedn{}}}} \\
    &\bsrule{EC-Seq}{\bigstep{s_0}{\sigma}{\store''} \qquad \bigstep{s_1}{\store''}{\excstore'}}
    {\bigstep{\defaultssaseq}{\store}{\excstore'}}\qquad
    \bsrule{EC-SeqAbrt}{\bigstep{s_0}{\sigma}{\exception}}
    {\bigstep{\defaultssaseq}{\store}{\exception}} \\
    &\bsrule{EC-IfT}{\bigstep{b}{\store}{\btrue}\qquad \bigstep{s_0}{\store}{\store''}
    \qquad \bigsteppl{\philist}{\dom\store}{\store''}{\store'}}
    {\bigstep{\defaultssaif}{\store}{\store'}}\\
    &\bsrule{EC-IfF}{\bigstep{b}{\store}{\bfalse}\qquad \bigstep{s_1}{\store}{\store''}
    \qquad \bigsteppr{\philist}{\dom\store}{\store''}{\store'}}
    {\bigstep{\defaultssaif}{\store}{\store'}}\\
    &\bsrule{EC-IfTAbrt}{\bigstep{b}{\store}{\btrue}\qquad \bigstep{s_0}{\store}{\exception}}
    {\bigstep{\defaultssaif}{\store}{\exception}} \qquad
    \bsrule{EC-IfFAbrt}{\bigstep{b}{\store}{\bfalse}\qquad \bigstep{s_1}{\store}{\exception}}
    {\bigstep{\defaultssaif}{\store}{\exception}} \\
    &\bsrule{EC-While}{\bigsteppl{\philist}{\dom\store}{\store}{\store''} \qquad
    \bigstepw{\defaultssawhile}{\dom\store}{\store''}{\excstore'}}
    {\bigstep{\defaultssawhile}{\store}{\excstore'}}\\
    &\bsrule{EC-Sink}{\bigstep{a}{\store}{\clean{}}}
    {\bigstep{\defaultsink}{\store}{\store}}\qquad
    \bsrule{EC-SinkAbrt}{\bigstep{a}{\store}{\tracked{}}}
    {\bigstep{\defaultsink}{\store}{\exception}}\\
\end{align*}

\subsubsection*{From dIMP to SSA-dIMP}
We show by example how the SSA construction works.
We consider the dIMP program shown in ~\autoref{fig:dimp-example}.
\begin{figure}[h]
    \begin{minted}[linenos]{pascal}
        I := X;
        Z := 1;
        while I > 0 do
            I := I - 1;
            if Z < 100 then
                Z := source Z * X
            else
                Z := Z
        ;
        sink Z
    \end{minted}    
    \caption{The example dIMP program}
    \label{fig:dimp-example}
\end{figure}
After transformation, an equivalent SSA-dIMP program is shown in~\autoref{fig:ssa-dimp-example}.
%TODO same figure for both

\begin{figure}[h]
    \begin{minted}[escapeinside=||,mathescape=true,linenos]{pascal}
        I0 := X;
        Z0 := 1;
        while [Z3 := |$\varphi$|(Z0, Z2), I2 := |$\varphi$|(I0, I1)]; I2 > 0 do
            I1 := I2 - 1;
            if Z3 < 100 then
                Z1 := source Z3 * X
            else
                Z1 := Z3
            ;[Z2 := |$\varphi$|(Z1, Z1)]
        ;
        sink Z3
    \end{minted}    
    \caption{The example program transformed to SSA-dIMP}
    \label{fig:ssa-dimp-example}
\end{figure}
The maps $\Gamma$ and $\Delta$ can be easily inferred, except for the while loop,
where the maps do not immediately can be inferred by using the rules.
Thus, as part of the transformation, we give the maps $\Delta$ and 
$\Delta_1$ as in rule $\textsc{SSA-While}$ for the while loop in the example program:
$\Delta = [Z_3 \mapsto \tint, I_2 \mapsto \tint]$ and $\Delta_1 = [I_3 \mapsto \tint, Z_2 \mapsto \tint]$.

This SSA form is obviously not unique, as the variable naming is not canonical.
Furthermore, the number of $\varphi$-nodes in the example program is minimal,
but no such assertions is made by the typing rules.

\subsubsection*{Correctness Results}
We state that an SSA-dIMP program together with the SSA type derivation
is in fact in SSA form --- each variable is defined only once.
Furthermore, the structure of the bigstep derivation trees imply that 
for every bigstep derivation, all variables are defined before they are used.
Proofs are omitted.
The first theorem is very easy to prove, and the second theorem is proved later in greater generality.

\begin{theorem}
    \label{thm:ssa-gamma-delta-disjoint}
    If $\typestep{\Gamma}{s}{\Delta}$ holds, then $\Gamma \cap \Delta = \emptyset$.
\end{theorem}

\begin{theorem}
    \label{thm:ssa-progress}
    If $\typestep{\Gamma}{s}{\Delta}$, $\dom\store = \Gamma$ and
    $\bigstep{s}{\store}{\store'}$, then $\store' = \store \dunion \store_0$
    and $\dom{\store_0} = \Delta$.
\end{theorem}

\subsection{Definition of Dataflow}
\begin{definition}[Program]
    A \emph{program} in SSA-dIMP is a command $s$ with sets $\Gamma, \Delta$ 
    with $\typestep{\Gamma}{s}{\Delta}$.
    Complex programs are expressed by using the recursive nature of the definition of statements.
\end{definition}

\begin{definition}[Initial Store]
    An \emph{initial store} is a store $\store$ such that $\initialstore{}$ holds, with 
    \begin{equation*}
        \initialstore{} \iff \forall X: \exists n: \store(X) = \clean[n]
    \end{equation*}
    This means that in an initial store all values are tagged as being clean.
\end{definition}

\begin{definition}[Dataflow]
    A tuple $(s, \store, \Gamma, \Delta)$ of a program $s$ and initial store $\store{}$ 
    with $\dom{\store{}} = \Gamma$ and $\initialstore$ 
    has \emph{dataflow from a source to a sink} if
    $\bigstep{s}{\store{}}{\exception}$ holds.
    This means that the program may abort.
\end{definition}

\begin{definition}[Dataflow Algorithm]
    A \emph{dataflow algorithm} $\A(s, \Gamma, \Delta)$ computes, given a program $s$
    and an SSA type derivation $\typestep{\Gamma}{s}{\Delta}$,
    whether there exists a store $\store{}$ 
    such that $\initialstore$ and $(s, \store, \Gamma, \Delta)$ has dataflow from a source to a sink.
    We require $\A(s, \Gamma, \Delta) \in \{\text{POSSIBLE\_FLOW}, \text{NO\_FLOW}\}$.
\end{definition}

\begin{definition}[Soundness]
    A dataflow algorithm is \emph{sound} if for all tuples $(s, \store, \Gamma, \Delta)$ with $\initialstore$ that
    have dataflow it holds that $\A(s, \Gamma, \Delta) = \text{POSSIBLE\_FLOW}$.
\end{definition}

\begin{definition}[Completeness]
    A dataflow algorithm is \emph{complete} if $\A(s, \Gamma, \Delta) = \text{NO\_FLOW}$
    implies that no store $\sigma$ with $\initialstore$ exists such that
    $(s, \store, \Gamma, \Delta)$ has dataflow.
\end{definition}

\begin{definition}[False Positive]
    A program $s$ for which there exists no store $\store$ such that 
    $(s, \initialstore, \Gamma, \Delta)$ with $\initialstore{}$ has dataflow,
     but for which $\A(s, \Gamma, \Delta) = \text{POSSIBLE\_FLOW}$ holds
    is called a \emph{false positive} of the algorithm.
\end{definition}
\begin{remark}
    In general, it is impossible to construct a dataflow algorithm that is both 
    sound and complete.
    In practice, a dataflow algorithm may be neither sound nor complete.
    However, in the theoretical setting of this chapter, we are interested in 
    sound dataflow algorithms.
    The \emph{trivially sound dataflow algorithm} $\A_0(s, \Gamma, \Delta) = \text{POSSIBLE\_FLOW}$ 
    is sound by definition, but not very interesting.
    We will not consider it further, but it is interesting to keep in mind,
    because it shows that just proving that a dataflow algorithm is sound does not
    mean it is useful.    
\end{remark}

\subsection{A General Setting For Flow Analysis}
\label{sec:df-theory}
In this section, we refine the type system presented for SSA to be viable as a
general-purpose constraint system.
This constraint system can then be solved via fixed-point iteration.
The system is designed in a way that it can be easily adapted to different analyses.
We instantiate this system to determine if programs have dataflow.

Let $\lattice$ be a poset.
Let $\join: \lattice \times \lattice \partialf \lattice$ be the (partially defined)
join operator on the poset.
We require that its domain is maximal, i.e. if $L_1, L_2 \in \lattice$ have a least upper
bound, it is $L_1 \join L_2$.
Let $\Gamma, \Delta: \Loc \partialf \lattice$ with finite domain.
Thus, we generalize the definition of $\Gamma$ and $\Delta$ here.
The previous definition can be recovered by setting $\lattice$ to be the one-point set.

For dataflow analysis, we have $\lattice = \{\lclean, \ltracked, \lunknown\}$.
Every variable encountered in the program is typed with an annotated type from \lattice.
Thus, it is either typed with $\lclean{}$ (the variable
is clean, i.e.\ does not contain a value originating from a dataflow source), $\lunknown{}$
(it is unknown whether the variable contains a value originating from a source) or $\ltracked{}$
(the variable contains a tracked value from a dataflow source).
We define the following poset structure, where the join operator is defined on the whole
poset:
\begin{figure}[h]
    \centering
    \begin{tikzpicture}
        \node (u) at (0, 1) {$\lunknown$};
        \node (c) at (-1, 0) {$\lclean$};
        \node (t) at (1, 0) {$\ltracked$};
        \draw[->] (c) -- (u);
        \draw[->] (t) -- (u);
    \end{tikzpicture}      
\end{figure}

We introduce the following judgements:
\\
Judgement \fbox{$\typestep{\Gamma}{a}{L}$}:
\begin{align*}
    &\trule{DF-Num}{}{\typestep{\Gamma}{\overline{n}}{\lclean}}\qquad
    \trule{DF-Loc}{}{\typestep{\Gamma}{X}{\Gamma(X)}}\qquad
    \trule{DF-Plus}{\typestep{\Gamma}{a_0}{L_0}\qquad\typestep{\Gamma}{a_1}{L_1}}
    {\typestep{\Gamma}{a_0+a_1}{\lclean}}\\
    &\trule{DF-Minus}{\typestep{\Gamma}{a_0}{L_0}\qquad\typestep{\Gamma}{a_1}{L_1}}
    {\typestep{\Gamma}{a_0-a_1}{\lclean}}\qquad
    \trule{DF-Times}{\typestep{\Gamma}{a_0}{L_0}\qquad\typestep{\Gamma}{a_1}{L_1}}
    {\typestep{\Gamma}{a_0 \times a_1}{\lclean}}\\
    &\trule{DF-Source}{\typestep{\Gamma}{a}{L}}
    {\typestep{\Gamma}{\source{a}}{\ltracked}}\\
\end{align*}
Judgement \fbox{$\typestmt{\Gamma}{b}$}:
\begin{align*}
    &\trule{DF-Bool}{}{\typestmt{\Gamma}{t}}\qquad
    \trule{DF-Eq}{\typestep{\Gamma}{a_0}{L_0}\qquad \typestep{\Gamma}{a_1}{L_1}}
    {\typestmt{\Gamma}{a_0 = a_1}}\\
    &\trule{DF-Leq}{\typestep{\Gamma}{a_0}{L_0}\qquad \typestep{\Gamma}{a_1}{L_1}}
    {\typestmt{\Gamma}{a_0 \leq a_1}}\qquad
    \trule{DF-Neg}{\typestmt{\Gamma}{b}}
    {\typestmt{\Gamma}{\neg b}}\qquad
    \trule{DF-And}{\typestmt{\Gamma}{b_0}\qquad\typestmt{\Gamma}{b_1}}
    {\typestmt{\Gamma}{b_0 \land b_1}}\\
\end{align*}
\\
Judgement \fbox{$\typestep{\Gamma;\Delta_0;\Delta_1}{\philist}{\Delta}$}:
\begin{align*}
    &\trule{DF-$\varphi$}{}{\typestep{\Gamma;\Delta_0;\Delta_1}{\defaultphilistn}{[X_1 \mapsto L_1, \ldots, X_n \mapsto L_n]}}\\
    &\text{(if for $i \in \{1, \ldots, n\} \exists L_i: X_i \notin \dom\Gamma \land (\Gamma \dunion \Delta_0)(Y_i) \leq L_i \land (\Gamma \dunion \Delta_1)(Z_i) \leq L_i
    \land \forall j \neq i: X_j \neq X_i$)}
\end{align*}
Note that this implies that this rule is only applicable if $Y_i$, $Z_i$ exist in the respective
domains of the partial functions, and if $L_i$ satisfying the constraints exists as well.\\
Judgement \fbox{$\typestep{\Gamma}{s}{\Delta}$}:
\begin{align*}
    &\trule{DF-Skip}{}{\typestep{\Gamma}{\skipcmd}{[]}}\qquad
    \trule{DF-Assign}{\typestep{\Gamma}{a}{L}}
    {\typestep{\Gamma}{\defaultstore}{[X \mapsto L]}} \text{($X \notin \dom\Gamma$)}\\
    &\trule{DF-Seq}{\typestep{\Gamma}{s_0}{\Delta_0}\qquad
    \typestep{\Gamma \cup \Delta_0}{s_1}{\Delta_1}}{\typestep{\Gamma}{\defaultssaseq}{\Delta_0 \cup \Delta_1}}\\
    &\trule{DF-If}{\typestmt{\Gamma}{b}\qquad\typestep{\Gamma}{s_0}{\Delta_0}\qquad \typestep{\Gamma}{s_1}{\Delta_1}\qquad
    \typestep{\Gamma;\Delta_0;\Delta_1}{\philist}{\Delta}}
    {\typestep{\Gamma}{\defaultssaif}{\Delta}}\\
    &\trule{DF-While}{\typestep{\Gamma;\emptyset;\Delta_1}{\philist}{\Delta}\qquad
    \typestmt{\Gamma \cup \Delta}{b}\qquad\typestep{\Gamma \cup \Delta}{s_0}{\Delta_1}}
    {\typestep{\Gamma}{\defaultssawhile}{\Delta}}\\
    &\trule{DF-Sink}{\typestep{\Gamma}{a}{L}}{\typestep{\Gamma}{\sink{a}}{[]}}\\
\end{align*}

\begin{definition}[Store-Matching]
    Let $\gamma:\lattice \to \P(\StoreDomain)$ by defined by
    \begin{align*}
        &\gamma:\lattice \to \P(\StoreDomain)\\
        &\gamma(L) = \begin{cases}
            \{\flagged{} \mid \flagged{} \in \StoreDomain \land f = \textbf{t}\} & \text{if }L = \ltracked\\
            \{\flagged{} \mid \flagged{} \in \StoreDomain \land  f = \textbf{c}\} & \text{if }L = \lclean\\
            \StoreDomain & \text{if }L = \lunknown\\
        \end{cases}
    \end{align*}
    A store $\store$ \emph{matches a description} $\Gamma: \Loc \partialf \lattice$ if it holds that
    \begin{equation*}
        \matches{\store}{\Gamma} \longeq \dom\store = \dom\Gamma \land \forall X: \sigma(X) \in \gamma(\Gamma(X))
    \end{equation*}
\end{definition}

First, we justify why the semantics contains disjoint unions when updating stores.
Then we prove some lemmas needed to establish the soundness of our dataflow analysis.
We omit the easy proofs.

\begin{theorem}
    \label{thm:gamma-delta-disjoint}
    If $\typestep{\Gamma}{s}{\Delta}$ holds, then $\dom\Gamma \cap \dom\Delta = \emptyset$.
\end{theorem}

\begin{lemma}
    \label{lem:gamma-arithm}
    Let $\E$ be a derivation of $\bigstep{a}{\sigma}{\flaggedn{}}$.
    If $\typestep{\Gamma}{a}{L}$ and $\matches{\store}{\Gamma}$, then $\flaggedn{} \in \gamma(L)$.
\end{lemma}

\begin{lemma}[Preservation-ish $\Phi$]
    \label{thm:preservation-phi}
    Let $\bigsteppl{\Phi}{\Theta}{\sigma''}{\sigma'}$ (or $\bigsteppr{\Phi}{\Theta}{\sigma''}{\sigma'}$),
    such that $\store'' = \store \dunion \store_0''\dunion \store{\_}$, $\matches{\store}{\Gamma}$, 
    $\matches{\store_0''}{\Delta_1}$ (or $\matches{\store_0''}{\Delta_2}$) and $\Theta = \dom\store$.
    Let $\typestep{\Gamma;\Delta_0;\Delta_1}{\Phi}{\Delta}$ by the DF rules.
    Then there exists a $\store_0$ such that $\store' = \store \dunion \store_0$
    and $\matches{\store_0}{\Delta}$.
\end{lemma}

\begin{theorem}[Preservation-ish]
    \label{thm:preservation}
    Suppose $\bigstep{s}{\store}{\store'}$ by $\E$, $\matches{\store}{\Gamma}$,
    and let $\typestep{\Gamma}{s}{\Delta}$ by $\D$ and the DF rules.
    Then there exists a $\store_0$ such that $\store' = \store \dunion \store_0$
    and $\matches{\store_0}{\Delta}$.
\end{theorem}
\begin{proof}
    We prove the theorem by induction over the derivation $\E$.\\
    \textbf{Case 1:}
    \textsc{EC-Skip}: Trivial, as $\Delta = \emptyset$.\\
    \textbf{Case 2:}
    \textsc{EC-Sink}: Trivial, as $\Delta = \emptyset$.\\
    \textbf{Case 3:}
    \begin{align*}
        &\E = \bsrule{EC-Seq}{\overset{\E_0}{\bigstep{s_0}{\sigma}{\store''}}
         \qquad \overset{\E_1}{\bigstep{s_1}{\store''}{\store'}}}
        {\bigstep{\defaultssaseq}{\store}{\store'}}\\
        &\D = \trule{DF-Seq}{\overset{\D_0}{\typestep{\Gamma}{s_0}{\Delta_0}}\qquad
        \overset{\D_1}{\typestep{\Gamma \cup \Delta_0}{s_1}{\Delta_1}}}
        {\typestep{\Gamma}{\defaultssaseq}{\Delta_0 \cup \Delta_1}}
    \end{align*}
    By assumption we have that $\matches{\store}{\Gamma}$.
    Thus, we can apply the IH on $\E_0$ with $\D_0$.
    With the IH we get $\store'' = \store \dunion \store''_0$ and $\matches{\store''_0}{\Delta_0}$
    with $\matches{\store''}{\Gamma \union \Delta_0}$.
    Thus, we can apply the IH on $\E_1$ with $\D_1$.
    Then we get that $\store' = \store'' \dunion \store'_0$ and $\matches{\store'_0}{\Delta_1}$.
    With that, we have $\store' = \store \dunion \store''_0 \dunion \store'_0$.
    We can set $\store_0 = \store''_0 \dunion \store'_0$, and we obviously 
    have that $\matches{\store_0}{\Delta_0 \union \Delta_1}$.\\
    \textbf{Case 4:}
    \begin{align*}
        &\E = \bsrule{EC-IfT}{
        \overset{}{\bigstep{b}{\store}{\btrue}}\qquad 
        \overset{\E_0}{\bigstep{s_0}{\store}{\store''}}
        \qquad 
        \overset{\E_1}{\bigsteppl{\philist}{\dom\store}{\store''}{\store'}}}
        {\bigstep{\defaultssaif}{\store}{\store'}}\\
        &\D = \trule{DF-If}{
        \overset{}{\typestmt{\Gamma}{b}}\qquad
        \overset{\D_0}{\typestep{\Gamma}{s_0}{\Delta_0}}\qquad \typestep{\Gamma}{s_1}{\Delta_1}\qquad
        \overset{\D_1}{\typestep{\Gamma;\Delta_0;\Delta_1}{\philist}{\Delta}}}
        {\typestep{\Gamma}{\defaultssaif}{\Delta}}
    \end{align*}
    By IH on $\E_0$ with $\D_0$, we get that $\store'' = \store \dunion \store''_0$ and 
    $\matches{\store''_0}{\Delta_0}$.
    The result then follows from~\autoref{thm:preservation-phi} on $\E_1$ with $\D_1$.\\
    \textbf{Case 5:}
    $\E$ uses \textsc{EC-IfF}: analogous.\\
    \textbf{Case 6:}
    \begin{align*}
        &\E = \bsrule{EC-Assign}{\overset{\E_0}{\bigstep{a}{\store}{\flaggedn{}}}}
        {\bigstep{\defaultstore}{\store}{\storeassign{\store}{X \mapsto \flaggedn{}}}}\\
        &\D = \trule{DF-Assign}{\overset{\D_0}{\typestep{\Gamma}{a}{L}}}
        {\typestep{\Gamma}{\defaultstore}{[X \mapsto L]}}
    \end{align*}
    We apply~\autoref{lem:gamma-arithm} on $\E_0$ with $\D_0$,
    and as $\flaggedn{} \in \gamma(L)$, we have that $\matches{[X \mapsto \flaggedn{}]}{[X \mapsto L]}$.
    The side-condition of \textsc{DF-Assign} implies that the union in the \textsc{EC-Assign}
    rule is actually disjoint.
    \\
    \textbf{Case 7:}\\
    \begin{align*}
        &\E = \bsrule{EC-While}{\overset{\E_0}{\bigsteppl{\philist}{\dom\store}{\store}{\store''}}\qquad
        \overset{\E_2}{\bigstepw{\defaultssawhile}{\dom\store}{\store''}{\excstore'}}}
        {\bigstep{\defaultssawhile}{\store}{\store'}}\\
        &\D = \trule{DF-While}{\overset{\D_0}{\typestep{\Gamma;\emptyset;\Delta_1}{\philist}{\Delta}}\qquad
        \overset{\D_1}{\typestmt{\Gamma \cup \Delta}{b}}\qquad
        \overset{\D_2}{\typestep{\Gamma \cup \Delta}{s_0}{\Delta_1}}}
        {\typestep{\Gamma}{\defaultssawhile}{\Delta}}
    \end{align*}
    By applying~\autoref{thm:preservation-phi} on $\E_0$ with $\D_0$, we get that
    $\store'' = \store \dunion \store_0$ and $\matches{\store_0}{\Delta}$.
    Thus, $\matches{\store''}{\Gamma \union \Delta}$.
    We prove the statement by an inner induction over the while derivation $\E^w$:
    \begin{claim}
        Let $\bigstepw{\defaultwhile}{\filter}{\store_w}{\store_w'}$ by $\E^w$,
         $\store_w = \store_w^0 \dunion \store_w^1$
        such that $\matches{\store_w^0}{\Gamma}$, $\Theta = \dom\store^0_w$, $\matches{\store_w^1}{\Delta}$
        and let $\typestep{\Gamma}{\defaultwhile}{\Delta}$ by $\D$ and the DF rules.
        Then we have that there exists a $\store_0$ such that $\store_w' = \store_w^0 \dunion \store_0$,
        and $\matches{\store_0}{\Delta}$.
    \end{claim}
    \begin{claimproof}
        \emph{Subcase 1:}
        \begin{align*}
            \E^w = \bsrule{EW-WhileF}{
            \overset{}{\bigstep{b}{\store_w}{\bfalse}}}
            {\bigstepw{\defaultssawhile}{\filter}{\store_w}{\store_w}}
        \end{align*}
        Trivial, taking $\store_0 = \store_w^1$.\\
        \emph{Subcase 2:}
        \begin{align*}
            \E^w = \bsrule{EW-WhileT}{\overset{}{\bigstep{b}{\store_w}{\btrue}} \enskip
            \overset{\E^w_0}{\bigstep{s_0}{\store_w}{\store''}} \enskip
            \overset{\E^w_1}{\bigsteppr{\philist}{\filter}{\store''}{\store'''}} \enskip
            \overset{\E^w_2}{\bigstepw{\defaultssawhile}{\filter}{\store'''}{\excstore'}}}
            {\bigstepw{\defaultssawhile}{\filter}{\store_w}{\store_w'}}
        \end{align*}
        By the outer IH on $\E^w_0$ with $\D_2$, we get that
        $\sigma'' = \sigma_w \dunion \sigma_w'' = \sigma_w^0 \dunion \sigma_w^1 \dunion \sigma_w''$
        and $\matches{\sigma_w''}{\Delta_1}$.
        By~\autoref{thm:preservation-phi} we get that $\store''' = \store_w^0 \dunion \store_0'''$ and
        $\matches{\store_0'''}{\Delta}$.
        With that we can apply the inner IH on $\E^2_w$ with $\D$ to get the result on $\sigma_w'$.
    \end{claimproof}
    We finish the proof by applying the claim on $\E^2$ with $\D$.
\end{proof}


We then have the following (general) soundness property of the dataflow analysis:
\begin{definition}[Safe Sink]
    Let $s, \Gamma$ be such that $s = \defaultsink$ and $\typestep{\Gamma}{s}{\Delta}$ (by $\D$).
    The sink invocation $s$ is \emph{safe}, if the derivation $\D$ has the shape
    \begin{align*}
        \D = \trule{DF-Sink}{\typestep{\Gamma}{a}{L}}{\typestep{\Gamma}{\defaultsink}{\emptyset}}
    \end{align*}
    with $L = \lclean$.
\end{definition}

\begin{lemma}
    \label{lem:df-soundness}
    Let $s$ with $\typestep{\Gamma_{ssa}}{s}{\Delta_{ssa}}$ be a program in SSA form and let $\store$ be a store
    with $\dom\store = \Gamma_{ssa}$ and $\matches{\store}{\Gamma}$.
    Let $\E$ be the bigstep derivation of $\bigstep{s}{\store}{\excstore{}'}$
    and $\typestep{\Gamma}{s}{\Delta}$ (by $\D$).
    Assume that all sinks in $\E$ (with $\D$) are safe.
    Then $\excstore{}' \neq \exception$.
\end{lemma}
\begin{proof}
    Proof by induction over the derivation $\E$.
    The cases \textsc{EC-Skip}, \textsc{EC-Sink}, \textsc{EC-Assign}, \textsc{EC-IfT} and \textsc{EC-IfF}
    are trivially clear, as they never evaluate to \exception.\\
    \textbf{Case 1:}
    \begin{align*}
        \E = \bsrule{EC-SinkAbrt}{\overset{}{\bigstep{a}{\store}{\tracked{}}}}
        {\bigstep{\defaultsink}{\store}{\exception}}\\
        \D = \trule{DF-Sink}{\overset{}{\typestep{\Gamma}{a}{L}}}
        {\typestep{\Gamma}{\sink{a}}{\emptyset}}
    \end{align*}
    However, because all sinks are safe, we have $L = \lclean$.
    Furthermore, as $\store : \Gamma$ this is a contradiction,
    so this case cannot happen.\\
    \textbf{Case 2:}
    \begin{align*}
        &\E = \bsrule{EC-SeqAbrt}{\overset{\E_0}{\bigstep{s_0}{\sigma}{\exception}}}
        {\bigstep{\defaultssaseq}{\store}{\exception}}\\
        &\D = \trule{DF-Seq}{\overset{\D_0}{\typestep{\Gamma}{s_0}{\Delta_0}}\qquad
        \overset{}{\typestep{\Gamma \cup \Delta_0}{s_1}{\Delta_1}}}
        {\typestep{\Gamma}{\defaultssaseq}{\Delta_0 \cup \Delta_1}}
    \end{align*}
    By IH on $\E_0$ with $\D_0$ we get that in $\bigstep{s_0}{\store}{\excstore{}'}$ 
    we have that $\excstore{}' \neq \exception$ holds, so this case cannot happen.
    The cases for \textsc{EC-IfTAbrt} and \textsc{EC-IfFAbrt} are proved analogous.\\
    \textbf{Case 3:}
    \begin{align*}
        &\E = \bsrule{EC-Seq}{\overset{\E_0}{\bigstep{s_0}{\sigma}{\store''}} \qquad 
        \overset{\E_1}{\bigstep{s_1}{\store''}{\excstore'}}}
        {\bigstep{\defaultssaseq}{\store}{\excstore'}}\\    
        &\D = \trule{DF-Seq}{\overset{\D_0}{\typestep{\Gamma}{s_0}{\Delta_0}}\qquad
        \overset{\D_1}{\typestep{\Gamma \cup \Delta_0}{s_1}{\Delta_1}}}
        {\typestep{\Gamma}{\defaultssaseq}{\Delta_0 \cup \Delta_1}}
    \end{align*}
    By~\autoref{thm:preservation} on $\E_0$ with $\D_0$ we get that $\matches{\store''}{\Gamma \cup \Delta_0}$.
    Then we apply the IH on $\E_1$ with $\D_1$ and get that $\excstore{}' \neq \exception$.\\
    \textbf{Case 4:}
    \begin{align*}
        &\E = \bsrule{EC-While}{\bigsteppl{\philist}{\dom\store}{\store}{\store''} \qquad
        \overset{\E_0}{\bigstepw{\defaultssawhile}{\dom\store}{\store''}{\excstore'}}}
        {\bigstep{\defaultssawhile}{\store}{\excstore'}}\\
        &\D = \trule{DF-While}{\overset{\D_0}{\typestep{\Gamma;\emptyset;\Delta_1}{\philist}{\Delta}}\qquad
        \typestmt{\Gamma \cup \Delta}{b}\qquad
        \overset{\D_1}{\typestep{\Gamma \cup \Delta}{s_0}{\Delta_1}}}
        {\typestep{\Gamma}{\defaultssawhile}{\Delta}}
    \end{align*}
    \begin{claim}
        Let $\store_w = \store_{w0} \dunion \store_{w1}$ with
        $\matches{\store_{w0}}{\Gamma}$ and $\matches{\store_{w1}}{\Delta}$,
         $\dom\Gamma = \filter$ and
        $\bigstepw{\defaultssawhile}{\filter}{\store_w}{\excstore{}'_w}$ by $\E^w$.
        Let $\typestep{\Gamma \union \Delta}{s_0}{\Delta_1}$ by $\D^w_0$
        and $\typestep{\Gamma;\emptyset;\Delta_1}{\philist}{\Delta}$ by $\D^w_1$.
        Then if all sinks in $\E^w$ are safe, $\excstore{}'_w \neq \exception$.
    \end{claim}
    \begin{claimproof}
        The case \textsc{EW-WhileF} is trivial, as it never evaluates to \exception.\\
        \emph{Subcase 1:}
        \begin{align*}
            \E^w = \bsrule{EW-WhileTAbrt}{\bigstep{b}{\store_w}{\btrue} \qquad 
            \overset{\E_0}{\bigstep{s_0}{\store_w}{\exception}}}
            {\bigstepw{\defaultssawhile}{\filter}{\store_w}{\exception}}
        \end{align*}
        By applying the outer IH on $\E_0$ with $\D^w_0$ we get that for 
        $\bigstep{s_0}{\store_w}{\excstore{}'}$ it actually holds that $\excstore{}' \neq \exception$,
        so this case cannot occur.\\
        \emph{Subcase 2:}
        \begin{align*}
            \E^w = \bsrule{EW-WhileT}{\overset{}{\bigstep{b}{\store_w}{\btrue}} \enskip
            \overset{\E^w_0}{\bigstep{s_0}{\store_w}{\store''}} \enskip
            \overset{\E^w_1}{\bigsteppr{\philist}{\filter}{\store''}{\store'''}} \enskip
            \overset{\E^w_2}{\bigstepw{\defaultssawhile}{\filter}{\store'''}{\excstore'_w}}}
            {\bigstepw{\defaultssawhile}{\filter}{\store_w}{\store_w'}}
        \end{align*}
        By~\autoref{thm:preservation} on $\E^w_0$ with $\D^w_0$, we get that $\store'' = \store_w \dunion \store''_0$
        and $\matches{\store''_0}{\Delta_1}$, thus $\matches{\store''}{\Gamma \union \Delta \dunion \Delta_1}$.
        By~\autoref{thm:preservation-phi} on $\E^w_1$ with $\D_1^w$ it follows that $\store''' = \store_{w0} \dunion \store'''_0$
        with $\matches{\store'''_0}{\Delta}$.
        Thus, $\matches{\store'''}{\Gamma \union \Delta}$ and we can apply the inner IH on $\E^w_2$.
        That implies that $\excstore{}'_w \neq \exception$.
    \end{claimproof}
    We apply the claim on $\E_0$ with $\D_0$ and $\D_1$.
\end{proof}

\begin{theorem}[Soundness of the DF rules]
    \label{thm:soundness-df}
    Let $(s, \Gamma_{ssa}, \Delta_{ssa})$ be a program in SSA form and let $\store$ be a store
    with $\dom\store = \Gamma_{ssa}$.
    Let $\E$ be the bigstep derivation of $\bigstep{s}{\store}{\excstore{}'}$.
    Furthermore, let $\Gamma: \Gamma_{ssa} \partialf \lattice$ be defined by
    \begin{align*}
        \Gamma(X) \mapsto \begin{cases}
            \lclean & \text{if $\exists n \in \Z: \store(X) = \clean$}\\
            \ltracked & \text{if $\exists n \in \Z: \store(X) = \tracked{}$}\\
        \end{cases}
    \end{align*}
    and let $\typestep{\Gamma}{s}{\Delta}$ (by $\D$).
    If all sinks in $\D$ are safe, then $\excstore{}' \neq \exception$.
\end{theorem}
\begin{proof}
    We have that $\matches{\store}{\Gamma}$ by construction.
    Thus, we can apply~\autoref{lem:df-soundness} to prove the theorem.
\end{proof}

We now describe a sound dataflow algorithm schema $\A(s, \Gamma_\text{ssa}, \Delta_\text{ssa})$.
This algorithm schema can be seen as a general specification for sound dataflow algorithms.
As soon as an algorithm satisfies this schema, it is sound.
In the next section, we will see an instantiation of this algorithm schema in QL.

First, define $\Gamma$ by
\begin{align*}
    \Gamma: &\Gamma_\text{ssa} \to \lattice\\
    &X \mapsto \lclean
\end{align*}
Then, compute a finite map $\Delta$ and a derivation tree $\D$ 
such that $\typestep{\Gamma}{s}{\Delta}$ (by $\D$) holds.
Output \text{POSSIBLE\_FLOW} if $\D$ contains a derivation of form
\begin{equation*}
    \trule{DF-Sink}{\typestep{\Gamma}{a}{\ltracked}}{\typestep{\Gamma}{\sink{a}}{\emptyset}}
\end{equation*}
or
\begin{equation*}
    \trule{DF-Sink}{\typestep{\Gamma}{a}{\lunknown}}{\typestep{\Gamma}{\sink{a}}{\emptyset}}
\end{equation*}

In the absence of these derivations, output \text{NO\_FLOW}.
\begin{corollary}
    The algorithm schema $\A(s, \Gamma_\text{ssa}, \Delta_\text{ssa})$ describes a sound dataflow algorithm.
\end{corollary}
