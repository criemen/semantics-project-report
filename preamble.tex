% !TEX root = main.tex
\documentclass[12pt,a4paper,bibliography=totoc]{scrartcl}
\usepackage[T1]{fontenc}
\usepackage[utf8]{inputenc}
\usepackage[english]{babel}
\usepackage{microtype}
\hyphenpenalty=750
\usepackage[style=alphabetic]{biblatex}
\bibliography{report}

\usepackage{amsmath}
\usepackage{libertine}
\usepackage[libertine]{newtxmath}
\usepackage{inconsolata}
\usepackage{soul}

\usepackage[shortlabels]{enumitem}
%\usepackage{tikz}
%\usetikzlibrary{positioning}
%\usetikzlibrary{shapes.geometric}
\usepackage{graphicx}
\usepackage{lastpage}
\usepackage{fancyvrb}
\usepackage{xfrac}
\usepackage{xspace}
\usepackage[usenames,dvipsnames]{xcolor}

\usepackage{bashful}

\usepackage{multirow}

\usepackage{framed}
\usepackage{environ}
\usepackage{multicol}
\newcommand{\TODO}[1]{}

\usepackage{minted}
\usemintedstyle{github}
\newminted{java}{linenos}
\newmintinline[java]{java}{linenos}
\providecommand*{\listingautorefname}{Listing}

\usepackage{csquotes}

%\usepackage{titlesec}

%\titlelabel{Question \thetitle:\ }

%\setlength{\parskip}{1ex}
%\setlength{\parindent}{0pt}
%\setlength{\parfillskip}{30pt plus 1 fil}

\usepackage{hyperref}
%\hypersetup{pdftitle={Take-home Exam in Advanced Programming, B1-2019/2020},
%            pdfsubject={Advanced Programming},
%            pdfauthor={Andrzej Filinski <andrzej@diku.dk>,
%              Ken Friis Larsen <kflarsen@diku.dk>},
%            pdfkeywords={exam},
%            pdfborder={0 0 0}}
%\fancyhf{}%
%\fancyhead[L]{\Large\textbf{Advanced Programming}}
%\fancyhead[R]{\Large\textbf{DIKU, B1-2019/2020}}
\def\headrulewidth{0mm}%
\def\footrulewidth{0mm}%
%\fancyfoot[C]{\thepage}
\setlength{\headheight}{22pt}

%\makeatletter
%{
% \catcode`\&=11
% \gdef\color@&spot#1#2{%
%  \ifdefined\spc@ir \else \spc@getir{\string\color@.}\fi
%  \c@lor@arg{#2}%
%  \edef#1{\spc@ir\space cs \spc@ir\space CS #2 sc #2 SC}}
%}

\urlstyle{rm}

% Abstand obere Blattkante zur Kopfzeile ist 2.54cm - 15mm
\setlength{\topmargin}{-15mm}

% number equations with sections
\numberwithin{equation}{section}

% setup to ignore spacing in verbatim environments in math mode, so formatting does not break
%\makeatletter
%\newcommand{\verbmathspace}{\let\FV@Space\space}
%\makeatother

% a verbatim environment to my liking
\DefineVerbatimEnvironment{codeverbatim}{Verbatim}{fontsize=\small,frame=single,gobble=8,commandchars=\\\{\},codes={\catcode`$=3\catcode`^=7\catcode`_=8\everymath\expandafter{\the\everymath\verbmathspace}}}
%codes={\catcode`$=3\catcode`^=7\catcode`_=8}}
\newenvironment{code}[1]{%
\VerbatimEnvironment
\begin{codeverbatim}[label=\textbf{#1}]%
}
{\end{codeverbatim}}


% einige Abkuerzungen
\newcommand\ddfrac[2]{\frac{\displaystyle #1}{\displaystyle #2}}
\newcommand*{\alg}[1]{\textup{\textrm{\textsc{#1}}}}
\newcommand*{\NP}{\textbf{NP}\xspace}
\DeclareMathOperator{\N}{\mathbb{N}}
\DeclareMathOperator{\R}{\mathbb{R}}
% Someone is already declaring \G
\let\G\undefined
\DeclareMathOperator{\G}{\mathbb{G}}
% Someone is already declaring \P
\let\P\undefined
\DeclareMathOperator{\P}{\mathbb{P}}
\DeclareMathOperator{\Z}{\mathbb{Z}}
%\DeclareMathOperator{\E}{\mathbb{E}}
\DeclareMathOperator{\F}{\mathcal{F}}
% Someone is already declaring \H
\let\H\undefined
\DeclareMathOperator{\H}{\mathcal{H}}
\DeclareMathOperator{\E}{\mathcal{E}}
\DeclareMathOperator{\T}{\mathcal{T}}
\DeclareMathOperator{\SD}{\mathcal{SD}}
% Someone is already declaring \S
\let\S\undefined
\DeclareMathOperator{\S}{\mathcal{S}}
\DeclareMathOperator{\A}{\mathcal{A}}
\DeclareMathOperator{\B}{\mathcal{B}}
\let\C\undefined
\DeclareMathOperator{\C}{\mathcal{C}}
\DeclareMathOperator{\ST}{\mathcal{ST}}
\DeclareMathOperator{\X}{\mathcal{X}}
\DeclareMathOperator{\Var}{\mathbb{V}ar}
\DeclareMathOperator{\xor}{\oplus}

% nicer enumerate
\renewcommand{\labelenumi}{(\arabic{enumi})}

% TODO delete
\usepackage{lipsum}
