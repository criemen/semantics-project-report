
% !TEX root = main.tex
\documentclass[12pt,a4paper]{scrartcl}
\usepackage{fontspec}
\usepackage[style=alphabetic]{biblatex}
\bibliography{report}

%\usepackage{lmodern}

\usepackage{fancyvrb}
\usepackage{framed}
\usepackage{abstract}
\usepackage{graphicx}
\usepackage{latexsym}
\usepackage{amsmath,amssymb}
\usepackage[amsmath,hyperref,amsthm,thmmarks]{ntheorem}

%\usepackage{mathpazo}
\usepackage{enumitem}
\usepackage{etoolbox}
\usepackage{libertine}

%\usepackage{longtable}
%\usepackage{pdflscape} % for 'landscape' environment, use pdflscape here to also rotate the pdf pages
%\usepackage{tabu}
\usepackage{aliascnt}
\usepackage{caption}%%Improves links to figures: Will show whole figure and not just the caption
%\usepackage{soul} %%for underlining categorynames

\usepackage[usenames,dvipsnames]{xcolor}
\usepackage[scale=0.8]{geometry}

%\usepackage{wasysym}
%\usepackage{stmaryrd}
%\usepackage{mathtools,leftidx}
\usepackage[shortcuts]{extdash} %for no breaking lines at hyphenation
\usepackage{mleftright}%%for better spacing aroung big brackets
\mleftright

\usepackage{xspace}
\usepackage[section]{placeins} % do not move figures across sections

% parts of the preamble which either need to be loaded very late or cannot be pre-compiled
\usepackage[english]{babel}
\usepackage{microtype}
\usepackage{minted}
\usepackage{csquotes}
\usepackage{hyperref}
\usepackage[nameinlink,capitalise]{cleveref}

\DeclareRobustCommand{\gray}[1]{%
\colorbox{gray!20}{$\displaystyle #1$}}

%\setmainfont
%     [ BoldFont       = texgyrepagella-bold.otf ,
%       ItalicFont     = texgyrepagella-italic.otf ,
%       BoldItalicFont = texgyrepagella-bolditalic.otf ]
%     {texgyrepagella-regular.otf}
\makeatletter
{
 \catcode`\&=11
 \gdef\color@&spot#1#2{%
  \ifdefined\spc@ir \else \spc@getir{\string\color@.}\fi
  \c@lor@arg{#2}%
  \edef#1{\spc@ir\space cs \spc@ir\space CS #2 sc #2 SC}}
}

\colorlet{colors}{blue}
\colorlet{colort}{cyan}
\colorlet{colorr}{purple!50!violet}
\colorlet{MyRed}{red}
\colorlet{MyGreen}{green!90!blue}

\urlstyle{rm}
%\usepackage{minted}
%\usepackage[epsilon,tstt]{backnaur}

% Abstand obere Blattkante zur Kopfzeile ist 2.54cm - 15mm
\setlength{\topmargin}{-15mm}

% We define some special styles, including setting the optional argument to definitions and theorems not in bold, and special claim and
% claim-proofs (like claim, but without numbering)
\makeatletter
\g@addto@macro\@floatboxreset\centering
\renewtheoremstyle{plain}%
{\item[\hskip\labelsep \theorem@headerfont ##1\ ##2\theorem@separator]}%
{\item[\hskip\labelsep \theorem@headerfont ##1\ ##2\ \textmd{(##3)}\theorem@separator]}
\newtheoremstyle{claimstyle}%
{\item[\hskip\labelsep \theorem@headerfont \underline{##1\ ##2\theorem@separator}]}%
{\item[\hskip\labelsep \theorem@headerfont \underline{##1\ ##2\ (##3)\theorem@separator}]}
\newtheoremstyle{claimproofstyle}%
{\item[\hskip\labelsep \theorem@headerfont \underline{##1\theorem@separator}]}%
{\item[\hskip\labelsep \theorem@headerfont \underline{##1\ (##3)\theorem@separator}]}
\makeatother


% Umgebungen für Definitionen, Sätze, usw.
% Es werden Sätze, Definitionen etc innerhalb einer Section mit
% 1.1, 1.2 etc durchnummeriert, ebenso die Gleichungen mit (1.1), (1.2) ..
\theoremseparator{.}
\theoremstyle{plain}
\newtheorem{theorem}{Theorem}[section]

\newaliascnt{lemma}{theorem}
\newtheorem{lemma}[lemma]{Lemma}
\aliascntresetthe{lemma}
\newaliascnt{proposition}{theorem}
\newtheorem{proposition}[proposition]{Proposition}
\aliascntresetthe{proposition}
\theoremstyle{definition}
\newaliascnt{definition}{theorem}
\newtheorem{definition}[definition]{Definition}
\aliascntresetthe{definition}
\newaliascnt{example}{theorem}
\newtheorem{example}[example]{Example}
\aliascntresetthe{example}
\newaliascnt{remark}{theorem}
\newtheorem{remark}[remark]{Remark}
\aliascntresetthe{remark}
\newaliascnt{notation}{theorem}
\newtheorem{notation}[notation]{Notation}
\aliascntresetthe{notation}


\providecommand*{\lemmaautorefname}{Lemma}
\providecommand*{\propositionautorefname}{Proposition}
\providecommand*{\definitionautorefname}{Definition}
\providecommand*{\exampleautorefname}{Example}
\providecommand*{\remarkautorefname}{Remark}


% Claim and Claimproofs need some special attention to look how I like them
\theoremseparator{:}
\theoremheaderfont{\upshape}
\theoremstyle{claimstyle}
\newtheorem{claim}{Claim}
\providecommand*{\lemmaautorefname}{Lemma}
% reset claim counters for each proof
\makeatletter
\@addtoreset{claim}{proof}
\makeatother
% setup autoref as it does not work for claims for some reason
\providecommand*{\claimautorefname}{Claim}


\theorembodyfont{\upshape}
\theoremstyle{claimproofstyle}
\theoremsymbol{\ensuremath{\blacksquare}}
\newtheorem{claimproof}{Proof}

% number equations with sections
\numberwithin{equation}{section}

% setup to ignore spacing in verbatim environments in math mode, so formatting does not break
\makeatletter
\newcommand{\verbmathspace}{\let\FV@Space\space}
\makeatother

\makeatletter
\newcommand{\vast}{\bBigg@{4}}
\newcommand{\Vast}{\bBigg@{5}}
\makeatother

% a verbatim environment to my liking
\DefineVerbatimEnvironment{codeverbatim}{Verbatim}{fontsize=\small,frame=single,gobble=8,commandchars=\\\{\},codes={\catcode`$=3\catcode`^=7\catcode`_=8\everymath\expandafter{\the\everymath\verbmathspace}}}
%codes={\catcode`$=3\catcode`^=7\catcode`_=8}}
\newenvironment{code}[1]{%
\VerbatimEnvironment
\begin{codeverbatim}[label=\textbf{#1}]%
}
{\end{codeverbatim}}


% einige Abkuerzungen
\newcommand\ddfrac[2]{\frac{\displaystyle #1}{\displaystyle #2}}
\newcommand*{\alg}[1]{\textup{\textrm{\textsc{#1}}}}
\newcommand*{\NP}{\textbf{NP}\xspace}
\DeclareMathOperator{\N}{\mathbb{N}}
\DeclareMathOperator{\R}{\mathbb{R}}
% Someone is already declaring \G
\let\G\undefined
\DeclareMathOperator{\G}{\mathbb{G}}
% Someone is already declaring \P
\let\P\undefined
\DeclareMathOperator{\P}{\mathbb{P}}
\DeclareMathOperator{\Z}{\mathbb{Z}}
%\DeclareMathOperator{\E}{\mathbb{E}}
\DeclareMathOperator{\F}{\mathcal{F}}
% Someone is already declaring \H
\let\H\undefined
\DeclareMathOperator{\H}{\mathcal{H}}
\DeclareMathOperator{\E}{\mathcal{E}}
\DeclareMathOperator{\T}{\mathcal{T}}
\DeclareMathOperator{\SD}{\mathcal{SD}}
% Someone is already declaring \S
\let\S\undefined
\DeclareMathOperator{\S}{\mathcal{S}}
\DeclareMathOperator{\A}{\mathcal{A}}
\DeclareMathOperator{\B}{\mathcal{B}}
\let\C\undefined
\DeclareMathOperator{\C}{\mathcal{C}}
\DeclareMathOperator{\ST}{\mathcal{ST}}
\DeclareMathOperator{\X}{\mathcal{X}}
\DeclareMathOperator{\Var}{\mathbb{V}ar}
\DeclareMathOperator{\xor}{\oplus}

% nicer enumerate
\renewcommand{\labelenumi}{(\arabic{enumi})}
%
%%New sections will start on odd pages
% \let\oldsection\section
% \def\section{\cleardoublepage\oldsection}



%\endofdump
